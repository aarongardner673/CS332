\section{Jay's attack}

Social engineering is a type of confidence trick or also might be described as a psychological manipulation of people to get them to perform some desired action or to
divulge sensitive information for the purpose of information gathering, fraud, or system access. It is often one of many steps in a more complex fraud scheme.
Social engineering has been used very often successfully among computer and information security professionals.
The techniques used for social engineering are based on specific attributes of human decision making known as cognitive biases. These biases can be thought of as bugs
in the human hardware, which can be exploited in various combinations that create opportunity for specific attack techniques, some of which will be covered in this
paper. 

Pretexting

Diversion theft

Phishing is one of the most used technique used to fraudulently attain someones private information. The most common vector used is an e-mail that looks to be from a
legitimate business, like a bank or credit card company, and is some form requesting varification of anything from passwords, ATM pin numbers, or credit card
information. It's usually accompanied by some warning of dire consequence if they are not provided. Another example of phishing is a link provided for the victim to
click on to verify their information by directing them to a spoofed website that looked exactly like the website their used to seeing. There they will be prompted by
a form to validate all their credentials for that site. It's fairly easy to make a website resemble a legitimate organization's site by mimicking the HTML code and
logos that the scam counted on the victim being tricked into thinking they were being contacted by that organization. The "phisher" basically shotguned or spammed
the e-mail to large groups of people in the hopes that some of the people had accounts with the fake website they made and they would actually be tricked into
responded with their desired credentials.

Spear phishing is similar to phishing with the difference being a highly customized email to only a few targeted users that have been well researched by the phisher.
Because of the extra effort in researching the victims and the more detail put into the e-mail yields a greater success rate of spear phishing attacks compared to
phishing attacks. To give an idea of the differences is success rates between phishing and spear phishing, phishing has a modest 5% average of all its massive amounts
of spammed e-mails to spear phishing's 50%. 

IVR or phone phishing

Water holing

Baiting

Quid pro quo

Tailgating