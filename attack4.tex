\section{Jay's Attack: Social Engineering}

Social engineering is a type of confidence trick or also might be described as a psychological manipulation of people to get them to perform some desired action or to
divulge sensitive information for the purpose of information gathering, fraud, or system access. It is often one of many steps in a more complex fraud scheme.
Social engineering has been used very often successfully among computer and information security professionals.
The techniques used for social engineering are based on specific attributes of human decision making known as cognitive biases. These biases can be thought of as bugs
in the human hardware, which can be exploited in various combinations that create opportunity for specific attack techniques, some of which will be covered in this
paper.

Phishing is one of the most used technique used to fraudulently attain someones private information. The most common vector used is an e-mail that looks to be from a
legitimate business, like a bank or credit card company, and is some form requesting varification of anything from passwords, ATM pin numbers, or credit card
information. It's usually accompanied by some warning of dire consequence if they are not provided. Another example of phishing is a link provided for the victim to
click on to verify their information by directing them to a spoofed website that looked exactly like the website their used to seeing. There they will be prompted by
a form to validate all their credentials for that site. It's fairly easy to make a website resemble a legitimate organization's site by mimicking the HTML code and
logos that the scam counted on the victim being tricked into thinking they were being contacted by that organization. The "phisher" basically shotguned or spammed
the e-mail to large groups of people in the hopes that some of the people had accounts with the fake website they made and they would actually be tricked into
responded with their desired credentials.

Spear phishing is similar to phishing with the difference being a highly customized email to only a few targeted users that have been well researched by the phisher.
Because of the extra effort in researching the victims and the more detail put into the e-mail yields a greater success rate of spear phishing attacks compared to
phishing attacks. To give an idea of the differences is success rates between phishing and spear phishing, phishing has a modest 5% average of all its massive amounts
of spammed e-mails to spear phishing's 50%.

Water holing is a more involved and sophisticated attack. The strategy is to capitalize on the trust users have in websites they regularly visit. The idea is that they
will do things like click on links without giving it much thought because of the trust they have for these regularly visited sites. The attacker uses this trust to set
a trap for the unwary victim at the favored watering hole. This technique has been used to gain access to very secure systems. This involves gathering information about
websites the targets often visit from a secure system. This survalence confirms that the targets visit the websites and that the secure system allows these visits. The
attack tests the sites for vulnerabilities to inject code that may infect the targets system with maleware. The injected code trap and maleware is taylored to the
specific target group and the specific systems they use. In time, one or more members of the target group will get infected and the attacker can gain access to the
secure system.

Baiting is a real world modern day Trojan horse that uses physical media and relies on the curiosity or greed of the victim. It's as simple as leaving a maleware-infected
device like a floppy disk, CD-ROM, or USB flash drive in locations where people will find them, like bathrooms, elevators, sidewalks, parking lots, etc. There will be
some sort of alluring feature like a big time corporate logo with a label saying "Executive Salary Summary Q4 2017". This should be enough to peak any of the targeted
victims afore mentioned shortcomings to pick the device up and using it to see what it has stored in it. All the attacker needs to gain access to the victim's PC, or
target company's internal computer network is for the victim to insert the device into the computer and the malware is installed.

Tailgating is when an attacker is seeking to gain entry to a restricted area secured by unattended, electronic access control like a RFID card. The attacker simply walks
in behind a person who has legitimate access to the area. The unknowing victim will follow common curtesy and will hold the door open for the attacker or the attacker may
themselves ask the employee to hold it open for them. If the victim asks for identification the attacker can just claim they lost the or forgot the appropriate identity
token, or even present a fake token. It's all in the convincing act of the attacker, and the gullibility of the victim.
